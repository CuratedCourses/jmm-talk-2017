\documentclass{chalkboard}

\begin{document}
\whitebackground

%%%%%%%%%%%%%%%%%%%%%%%%%%%%%%%%%%%%%%%%%%%%%%%%%%%%%%%%%%%%%%%%
% TITLE SLIDE
\clearbackgroundpicture
\begin{frame}[nofills]
  \vspace{5ex}

  \large
  \scaletowidth{\textwidth}{Using \textbf{CuratedCourses} to} \\[1ex]
  \scaletowidth{\textwidth}{match OER to other OER}

  \vfill
  
  \begin{columns}
    \begin{column}{0.39\textwidth}
      \color{osugray}
      \large
      \textbf{MAA Session} \\
      \textbf{The Advancement of OER} \\
      9:00am January 12, 2018 \\
      Room 31C % , San Diego Convention Center
    \end{column}

    \begin{column}{0.6\textwidth}
      \large
      \textbf{Petra Bonfert-Taylor}, Dartmouth College \\
      \textbf{Sarah Eichhorn}, University of Texas at Austin  \\
      \textbf{David Farmer}, American Institute of Mathematics  \\
      \textbf{Jim Fowler}, The Ohio State University  \\
    \end{column}

\end{columns}
  
\end{frame}

%%%%%%%%%%%%%%%%%%%%%%%%%%%%%%%%%%%%%%%%%%%%%%%%%%%%%%%%%%%%%%%%
% joint work
\begin{frame}
  \huge

  Many content creators have contributed \\
  \quad their excellent work to make this project possible.

  \vfill

  NSF support under DUE--1505246. \\

  \textcolor{gray}{Any opinions, findings, and conclusions or
    recommendations expressed in this material are those of the
    author(s) and do not necessarily reflect the views of the National
    Science Foundation.}

\end{frame}

%%%%%%%%%%%%%%%%%%%%%%%%%%%%%%%%%%%%%%%%%%%%%%%%%%%%%%%%%%%%%%%%
% user stories
\begin{frame}[nofills]
\vfill

\scaletowidth{\textwidth}{User Stories} \\[24pt]

\vfill
\end{frame}

%%%%%%%%%%%%%%%%%%%%%%%%%%%%%%%%%%%%%%%%%%%%%%%%%%%%%%%%%%%%%%%%
% harm: the problem
\begin{frame}
  \huge 

  ``As an instructor, I want to quickly find \\
  \quad a good video \\
  \quad\quad related to the theorem I'll be discussing today.''

  \vfill

  ``As a student, I want to find \\
  \quad some additional practice problems \\
  \quad\quad for this section of my textbook.''
\end{frame}

\begin{frame}
  \huge

  Find resources which are \\
  \quad available online, \\
  \quad released under an open license, \\
  \quad broadly useful, \\
  \quad high quality.

  \vfill

  One ``solution'' for instructors is \\
  \quad to create all the content.

\end{frame}


%%%%%%%%%%%%%%%%%%%%%%%%%%%%%%%%%%%%%%%%%%%%%%%%%%%%%%%%%%%%%%%%
% philosophical overview: creation versus curation
\begin{frame}[nofills]
\vfill

\scaletowidth{\textwidth}{curation} \\[24pt]

\begin{center}\color{gray}{\scaletowidth{0.35\textwidth}{versus}}\end{center}

\scaletowidth{\textwidth}{creation}

\vfill

\end{frame}

\begin{frame}
  \huge
  \vfill

  To curate existing resources, \\
  \quad a website where resources are\\
  \quad\quad collected and  \\
  \quad\quad linked to topics/outcomes and \\
  \quad\quad reviewed \\
  \quad would be helpful.

  \vfill

\end{frame}

%%%%%%%%%%%%%%%%%%%%%%%%%%%%%%%%%%%%%%%%%%%%%%%%%%%%%%%%%%%%%%%%
% user stories
\begin{frame}[nofills]
\vfill

\scaletowidth{\textwidth}{Doesn't this} \\[24pt]

\scaletowidth{\textwidth}{already exist?}

\vfill
\end{frame}

%%%%%%%%%%%%%%%%%%%%%%%%%%%%%%%%%%%%%%%%%%%%%%%%%%%%%%%%%%%%%%%%
% INHERENCY: what the status quo is providing isn't good enough
%
% these screenshots demonstrate the lack of reasonable tags 
%
\setbackgroundpicturewhite{others/merlot-1.png}
\begin{frame}
\end{frame}

\setbackgroundpicturewhite{others/merlot-2.png}
\begin{frame}
\end{frame}

\setbackgroundpicturewhite{others/oer-commons.png}
\begin{frame}
\end{frame}

\setbackgroundpicturewhite{others/curriki-khan-academy.png}
\begin{frame}
\end{frame}

\setbackgroundpicturewhite{others/open-math-notes.png}
\begin{frame}
\end{frame}

\clearbackgroundpicture

\begin{frame}
  \vfill

  \huge 

  Existing platforms \\
  \quad operate at a level of granularity,  \\
  \quad for both \\
  \quad\quad content and tags, \\
  \quad coarser than a \\
  \quad\quad topic or learning outcome.

  \vfill
\end{frame}

\setbackgroundpicturewhite{workflow/textbook.png}
\begin{frame}
\end{frame}
\setdarkbackgroundpicturewhite{workflow/textbook.png}
\begin{frame}[nofills]
\vfill

\scaletowidth{\textwidth}{Align OER to Textbooks} \\
\scaletowidth{\textwidth}{(and textbooks to each other)}

\vfill
\end{frame}
\clearbackgroundpicture

\setbackgroundpicturewhite{workflow/tag-list.png}
\begin{frame}
\end{frame}
\setdarkbackgroundpicturewhite{workflow/tag-list.png}
\begin{frame}[nofills]
\vfill

\scaletowidth{\textwidth}{Find relevant content} \\[16pt]
\scaletowidth{\textwidth}{for what you're teaching today} \\[16pt]

\vfill
\end{frame}
\clearbackgroundpicture

\begin{frame}[nofills]
\huge
\vfill
\vfill

Collaboratively create tags for topics and outcomes.

\vfill

Invite the community to add resources \\
\quad linked to these tags, and therefore \\
\quad aligned to popular textbooks.

\vfill

Review the submitted resources.

\vfill
\vfill
\end{frame}

\setbackgroundpicturewhite{workflow/choose-format.png}
\begin{frame}
\end{frame}

\setbackgroundpicturewhite{workflow/describe1.png}
\begin{frame}
\end{frame}

\setbackgroundpicturewhite{workflow/describe2.png}
\begin{frame}
\end{frame}

\setbackgroundpicturewhite{workflow/describe3.png}
\begin{frame}
\end{frame}

\setbackgroundpicturewhite{workflow/tags.png}
\begin{frame}
\end{frame}

\clearbackgroundpicture


\settallbackgroundpicturewhite{workflow/approval.png}
\begin{frame}
\end{frame}
\clearbackgroundpicture

\begin{frame}
  \huge

  How is this sustainable?

  \vfill

  Invisible metadata rots, \\
  \quad therefore, \\
  we must make our metadata visible.

\end{frame}

\settallbackgroundpicturewhite{workflow/right-margin.png}
\begin{frame}
\end{frame}
\clearbackgroundpicture

\begin{frame}
  \huge

  On your webpage, \\
  \quad include a \texttt{<span>} with a topic/outcome tag and \\
  \quad a \texttt{<script>} pointing to \texttt{curatedcourses.org}

  \vfill

  CuratedCourses then adds to your page \\
  \quad the name of the outcome and \\
  \quad links to high-quality relevant resources.
\end{frame}

\begin{frame}
  \huge

  \vfill

  The plan has been to
  \begin{description}[empower]
  \item[create] or \textit{find} content;
  \item[curate] or \textit{review} to ensure quality and alignment;
  \item[empower] instructors to \textit{promote} OER.
  \end{description}

  \vfill
\end{frame}

%%%%%%%%%%%%%%%%%%%%%%%%%%%%%%%%%%%%%%%%%%%%%%%%%%%%%%%%%%%%%%%%
% THANK YOU
 \clearbackgroundpicture
 \begin{frame}[label=thanks,nofills]
   \vfill
   \begin{center}
   \Huge
    \scalebox{1.5}{\textbf{Thank You}}
   \end{center}
   \vfill
   \vfill
   \includegraphics[width=1in]{images/cc-logo.pdf}\hfill\footnotesize\scalebox{0.75}{\textcolor{ccgray}{Licensed for reuse under a Creative Commons BY-SA License}}
   \null
   %\vspace{12pt}
   \null
 \end{frame}



%Sad XML Truth No. 3: Interoperability isn�t an engineering issue,
%it�s a business issue.  Creating the Web -- HTTP plus HTML -- was
%probably the last instance where standards of global importance were
%designed and implemented without commercial interference. Standards
%have become too important as competitive tools to leave them where
%they belong, in the hands of engineers. Incompatibility doesn't exist
%because companies can't figure out how to cooperate with one
%another. It exists because they don�t want to cooperate with one another.

\end{document}
