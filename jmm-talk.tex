\documentclass{chalkboard}

\begin{document}
\whitebackground

%%%%%%%%%%%%%%%%%%%%%%%%%%%%%%%%%%%%%%%%%%%%%%%%%%%%%%%%%%%%%%%%
% TITLE SLIDE
\clearbackgroundpicture
\begin{frame}[nofills]
  \vspace{5ex}

  \large
  \scaletowidth{\textwidth}{\textbf{Find, Review, Promote}} \\[1ex]
  \scaletowidth{\textwidth}{\textbf{CuratedCourses aligns OER to the course}}

  \vfill
  
  \color{osugray}

  \begin{columns}
    \begin{column}{0.49\textwidth}
      \large
      \vspace{2ex}

      \textbf{MAA Session} \\
      \textbf{The Advancement of OER} \\
      1:00pm January 7, 2017 \\
      A702, Atrium Level, Marriott Marquis

    \end{column}

    \hfill
    \begin{column}{0.5\textwidth}
      \large
      \textsf{\textbf{Jim Fowler}} \\
      \textsf{The Ohio State University} \\
      \textsf{Department of Mathematics} \\[2ex]

      \textsf{\textbf{David Farmer}} \\
      \textsf{American Institute of Mathematics} \\
    \end{column}

\end{columns}
  
\end{frame}

%%%%%%%%%%%%%%%%%%%%%%%%%%%%%%%%%%%%%%%%%%%%%%%%%%%%%%%%%%%%%%%%
% joint work
\begin{frame}
  \huge

  Collaboration with \\
  \quad Petra Bonfert-Taylor \textcolor{gray}{(Dartmouth College)} and \\
  \quad Sarah Eichhorn \textcolor{gray}{(University of California, Irvine)}

  \vfill

  NSF support under DUE--1505246. \\

  \textcolor{gray}{Any opinions, findings, and conclusions or
    recommendations expressed in this material are those of the
    author(s) and do not necessarily reflect the views of the National
    Science Foundation.}

\end{frame}

%%%%%%%%%%%%%%%%%%%%%%%%%%%%%%%%%%%%%%%%%%%%%%%%%%%%%%%%%%%%%%%%
% harm: the problem
\begin{frame}
  \huge 

  ``As an instructor, I want to quickly find \\
  \quad a good video \\
  \quad\quad related to the theorem I'll be discussing today.''

  \vfill

  ``As a student, I want to find \\
  \quad some additional practice problems \\
  \quad\quad for this section of my textbook.''
\end{frame}

\begin{frame}
  \huge

  Find resources which are \\
  \quad available online, \\
  \quad released under an open license, \\
  \quad broadly useful, \\
  \quad high quality.

  \vfill

  One ``solution'' for instructors is \\
  \quad to create our own content.

\end{frame}


%%%%%%%%%%%%%%%%%%%%%%%%%%%%%%%%%%%%%%%%%%%%%%%%%%%%%%%%%%%%%%%%
% philosophical overview: creation versus curation
\begin{frame}[nofills]
\vfill

\scaletowidth{\textwidth}{curation} \\[24pt]

\begin{center}\color{gray}{\scaletowidth{0.35\textwidth}{versus}}\end{center}

\scaletowidth{\textwidth}{creation}

\vfill

\end{frame}

%%%%%%%%%%%%%%%%%%%%%%%%%%%%%%%%%%%%%%%%%%%%%%%%%%%%%%%%%%%%%%%%
% INHERENCY: what the status quo is providing isn't good enough
%
% these screenshots demonstrate the lack of reasonable tags 
%
\setbackgroundpicturewhite{others/merlot-1.png}
\begin{frame}
\end{frame}

\setbackgroundpicturewhite{others/merlot-2.png}
\begin{frame}
\end{frame}

\setbackgroundpicturewhite{others/oer-commons.png}
\begin{frame}
\end{frame}

\setbackgroundpicturewhite{others/curriki-khan-academy.png}
\begin{frame}
\end{frame}

\clearbackgroundpicture

\begin{frame}
  how does one tag their content?

  BADBAD
\end{frame}

\begin{frame}
against what taxonomy?  

aligned to textbooks

  BADBAD
\end{frame}

\begin{frame}
how do reviewers submit a review?  

  BADBAD
\end{frame}

\begin{frame}
what data would constitute a "review" of a resource?

  BADBAD
\end{frame}

\setbackgroundpicturewhite{workflow/choose-format.png}
\begin{frame}
\end{frame}

\setbackgroundpicturewhite{workflow/describe1.png}
\begin{frame}
\end{frame}

\setbackgroundpicturewhite{workflow/describe2.png}
\begin{frame}
\end{frame}

\setbackgroundpicturewhite{workflow/describe3.png}
\begin{frame}
\end{frame}

\setbackgroundpicturewhite{workflow/tags.png}
\begin{frame}
\end{frame}

\clearbackgroundpicture

\begin{frame}
how does the platform decide what content gets shown?  

  BADBAD
\end{frame}

\settallbackgroundpicturewhite{workflow/approval.png}
\begin{frame}
\end{frame}
\clearbackgroundpicture


%%%%%%%%%%%%%%%%%%%%%%%%%%%%%%%%%%%%%%%%%%%%%%%%%%%%%%%%%%%%%%%%
% THANK YOU
 \clearbackgroundpicture
 \begin{frame}[label=thanks,nofills]
   \vfill
   \begin{center}
   \Huge
    \scalebox{1.5}{\textbf{Thank You}}
   \end{center}
   \vfill
   \vfill
   \includegraphics[width=1in]{images/cc-logo.pdf}\hfill\footnotesize\scalebox{0.75}{\textcolor{ccgray}{Licensed for reuse under a Creative Commons BY-SA License}}
   \null
   %\vspace{12pt}
   \null
 \end{frame}



%Sad XML Truth No. 3: Interoperability isn�t an engineering issue,
%it�s a business issue.  Creating the Web -- HTTP plus HTML -- was
%probably the last instance where standards of global importance were
%designed and implemented without commercial interference. Standards
%have become too important as competitive tools to leave them where
%they belong, in the hands of engineers. Incompatibility doesn't exist
%because companies can't figure out how to cooperate with one
%another. It exists because they don�t want to cooperate with one another.

\end{document}
